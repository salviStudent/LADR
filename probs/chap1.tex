\documentclass{article}
\usepackage[utf8]{inputenc}
\usepackage{amsmath}
\usepackage{amsfonts}
\usepackage{amssymb}
\usepackage{titlesec}
\usepackage[left=4cm,right=4cm,top=4cm,bottom=4cm]{geometry}

\usepackage{fancyhdr}
\usepackage{mathtools}
\titlelabel{\thetitle\enspace}

\begin{document}
\title{Linear Algebra Done Right \protect\\
  Solutions To Chapter 1 Exercises}
\author{Vladimir Guevara-Gonzalez}
\maketitle
\thispagestyle{fancy}
\begin{enumerate}
\item \( \dfrac{1}{a+bi} = c+di\) \newline
  \(1 = (c+di)(a+bi)\) \newline
  \(1 = ac + adi + bci + bd(i)^{2}\) \newline
  \(1 = ac - bd + (ad+bc)i\) \newline
  \newline
  \(ac-bd=1\) \newline
  \(ad + bc = 0,  \quad   c  = \dfrac{-ad}{b}\)\newline
  \(\dfrac{a(-ad)}{b} - bd = 1\) \newline
  \(\dfrac{-a^{2}d}{b} - bd = 1\) \newline
  \((-a^{2}-b^{2})d = b\) \newline
  \(d = \dfrac{b}{-a^{2}-b^{2}} = \dfrac{-b}{a^{2}+b^{2}} \; \text{thus},
  \quad c = \dfrac{-ab}{(-a^{2}-b^{2})b} = \dfrac{a}{a^{2}+b^{2}}\) \newline
  \(d = \dfrac{-b}{a^{2}+b^{2}} \quad c = \dfrac{a}{a^{2}+b^{2}}\)
\item
  To show \(\dfrac{-1+3\sqrt{3}i}{2} \) is a cube root of $1$ we can just take
  \(\dfrac{-1+3\sqrt{3}i}{2} \) and cube it. \newline \newline
  \(\biggl(\dfrac{-1+3\sqrt{3}i}{2}\biggr)^{2}  = \dfrac{1-2\sqrt{3}i-3}{4}\) \newline \newline
  \( \dfrac{(1-2\sqrt{3}i-3)}{4} \dfrac{(-1+3\sqrt{3}i)}{2} =
  \dfrac{(-1 + 3\sqrt{3}i + 9 -3\sqrt{3}i)}{8} = \dfrac{8}{8} = 1, \;
  \text{which is the result }\) \newline
  \(\text{we were looking for. Thus }\dfrac{-1+3\sqrt{3}i}{2} \) \text{is a cubed root of 1.}

\item
  \(\text{Let v}  \in V \text{where V is a vector space, then} -\text{v} \text{ is the additive
    inverse of v.} \) \newline
  Consider \( -(-\text{v})\text{. By the properties of of vector spaces }
  -(-\text{v}) = 0 + -(-\text{v})\)\newline
  \(= \text{v} + -\text{v} + -(-\text{v}) = \text{v, as desired.}\)

\end{enumerate}
\end{document}
